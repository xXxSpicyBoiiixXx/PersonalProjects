% --------------------------------------------------------------
% This is all preamble stuff that you don't have to worry about.
% Head down to where it says "Start here"
% --------------------------------------------------------------
 
\documentclass[12pt]{article}
 
\usepackage[margin=1in]{geometry} 
\usepackage{amsmath,amsthm,amssymb}
 
\newcommand{\N}{\mathbb{N}}
\newcommand{\Z}{\mathbb{Z}}
 
\newenvironment{theorem}[2][Theorem]{\begin{trivlist}
\item[\hskip \labelsep {\bfseries #1}\hskip \labelsep {\bfseries #2.}]}{\end{trivlist}}
\newenvironment{lemma}[2][Lemma]{\begin{trivlist}
\item[\hskip \labelsep {\bfseries #1}\hskip \labelsep {\bfseries #2.}]}{\end{trivlist}}
\newenvironment{exercise}[2][Exercise]{\begin{trivlist}
\item[\hskip \labelsep {\bfseries #1}\hskip \labelsep {\bfseries #2.}]}{\end{trivlist}}
\newenvironment{problem}[2][Problem]{\begin{trivlist}
\item[\hskip \labelsep {\bfseries #1}\hskip \labelsep {\bfseries #2.}]}{\end{trivlist}}
\newenvironment{question}[2][Question]{\begin{trivlist}
\item[\hskip \labelsep {\bfseries #1}\hskip \labelsep {\bfseries #2.}]}{\end{trivlist}}
\newenvironment{corollary}[2][Corollary]{\begin{trivlist}
\item[\hskip \labelsep {\bfseries #1}\hskip \labelsep {\bfseries #2.}]}{\end{trivlist}}
 
\begin{document}
 
% --------------------------------------------------------------
%                         Start here
% --------------------------------------------------------------
 
\title{Assignment 10}%replace X with the appropriate number
\author{Md Ali\\ %replace with your name
Abstract Algebra MATH 4307} %if necessary, replace with your course title
 
\maketitle
 
\begin{problem}{2.3.1} %You can use theorem, exercise, problem, or question here.  Modify x.yz to be whatever number you are proving
If $A$, $B$ are subgroups of $G$, show that $A \cap B$ is a subgroup of $G$.
\end{problem}
 
\begin{proof}
We must prove that $A \cap B$ is a subgroup of $G$. \\ \\
$A \cap B$ is a non-empty subset of $G$ since $e$ belongs to both $A$ and $B$, e being the identity element.\\ \\
Let $a,b \in A \cap B$ \\ \\
\centerline{Then $a,b \in A$ and $a,b \in B$}
\centerline{Since $A$ is a subgroup, $a,b \in A \rightarrow ab^{-1} \in A$}
\centerline{Since $B$ is a subgroup, $a,b \in B \rightarrow ab^{-1} \in B$} \\ \\
\centerline{Therefore, $a \in A \cap B$, $b \in A \cap B \rightarrow ab^{-1} \in A \cap B$} \\ \\
\centerline{\fbox{Hence $A \cap B$ is a subgroup of $G$.}}
\end{proof}

\begin{problem}{2.3.2}
What is the cyclic subgroup of $\mathbb{Z}$ generated by $-1$ under $+$?
\end{problem}

\begin{proof}
Note that $\mathbb{Z}$ is an infinite cyclic group. We claim that, if $H$ is a cyclic subgroup of $\mathbb{Z}$ generated by $-1$, then $H= \mathbb{Z}$. \\ \\
Let $x$ be an arbitrary element of $\mathbb{Z}$ \\ \\
\centerline{$m=1$, $m=(-1)(-m) = (-1)m_1$, where $m_1 \in \mathbb{Z}$} \\ \\
Since $m$ is an arbitrary element of $\mathbb{Z}$, it follows that \\ \\
\centerline{$\mathbb{Z} = \left<-1\right> = H$} \\ \\
\centerline{\fbox{Hence the cyclic subgroup of $\mathbb{Z}$ generated by $-1$ is $\mathbb{Z}$ itself. Hence result}} \\ \\
\end{proof}

\begin{problem}{2.3.4}
Verify that $Z(G)$, the center of $G$, is a subgroup of $G$.
\end{problem}

\begin{proof}
The centre $Z(G) = \{x \in G: xg = gx$  $\forall$  $g \in G\}$ is a non-empty subset of $G$ because $e \in Z(G)$, where $e$ is the identity element. \\ \\
Let $p,q \in Z(G)$ \\ \\
\centerline{We get $pg = gp$, $qg=gq$,  $\forall$  $g \in G$}
\centerline{$(pq)g=p(qg)=(pg)q=(gp)q=g(pq)$, $\forall$ $g \in G$} \\ \\
This shows that $pq \in Z(G)$, hence \\ \\
\centerline{$p \in Z(G), q \in Z(G) \rightarrow pq \in Z(G)$} \\ \\
Let $p\in Z(G)$, then \\ \\
\centerline{$pg=gp$ $\forall$ $g \in G$} \\ \\
\centerline{$gp^{-1}=p^{-1}(pg)p^{-1}=p^{-1}(gp)p^{-1}=p^{-1}g$, $\forall$ $g\in G$} \\ \\
This show that $p^{-1} \in Z(G)$ \\ \\
Hence, \\
\centerline{$p \in Z(G) \rightarrow p^{-1} \in Z(G)$} \\ \\
\centerline{\fbox{Hence from above $Z(G)$ is a subgroup of $G$}} \\ \\
\end{proof}
 
 \begin{problem}{2.3.6}
Show that $a \in Z(G)$ if and only if $C(a) = G$
\end{problem}

\begin{proof}
Let's take $Z(G) := \{g\in G | gx = xg$ $\forall$ $x\in G\}$ is the centre of $G$. \\ \\
Let $a \in Z(G)$, then \\ \\
\centerline{$ax=xa$ $\forall$ $x\in G$} \\ \\
We get that \\ \\
\centerline{$x \in C(a)$ $\forall$ $x\in G$} \\ \\
\centerline{Hence $G = C(a)$} \\ \\
Conversely, let's make $C(a) = G$ \\ \\
\centerline{$ax=xa$ $\forall$ $x\in G$} \\ \\
This implies that $a\in Z(G)$ \\ \\
\centerline{\fbox{Hence, we can see that $a \in Z(G)$ if and only if $C(a) = G$}}
\end{proof}

\begin{problem}{2.3.8}
If $G$ is an abelian group and if $H = \{ a \in G | a^2 =e \}$, show that $H$ is a subgroup of $G$
\end{problem}

\begin{proof}
We can see that $H$ is a non-empty subset of $G$, since $e^2 = e \rightarrow e \in H$. \\ \\
Let $x,y \in H$ \\ \\ 
\centerline{$x^2 = e$ and $y^2 = e$} \\ \\
Since $G$ is abelian we get that $xy = yx$ \\ \\
Now let's look at  \\ \\
\centerline{$(xy)^2=(xy)(xy)=x(yx)y=x(xy)y=x^2y^2=e \cdot e=e$} \\ \\
This makes that $x,y \in H \rightarrow xy \in H$ \\ \\
\centerline{$(x^{-1})^2=(x^2)^{-1}=e^{-1}=e$} \\ \\
\centerline{Hence, $x \in H \rightarrow x^{-1} \in H$} \\ \\
\centerline{\fbox{Hence $H$ is a subgroup of $G$}}


\end{proof}



% --------------------------------------------------------------
%     You don't have to mess with anything below this line.
% --------------------------------------------------------------
 
\end{document}