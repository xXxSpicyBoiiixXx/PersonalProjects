% --------------------------------------------------------------
% This is all preamble stuff that you don't have to worry about.
% Head down to where it says "Start here"
% --------------------------------------------------------------
 
\documentclass[12pt]{article}
 
\usepackage[margin=1in]{geometry} 
\usepackage{amsmath,amsthm,amssymb}
 
\newcommand{\N}{\mathbb{N}}
\newcommand{\Z}{\mathbb{Z}}
 
\newenvironment{theorem}[2][Theorem]{\begin{trivlist}
\item[\hskip \labelsep {\bfseries #1}\hskip \labelsep {\bfseries #2.}]}{\end{trivlist}}
\newenvironment{lemma}[2][Lemma]{\begin{trivlist}
\item[\hskip \labelsep {\bfseries #1}\hskip \labelsep {\bfseries #2.}]}{\end{trivlist}}
\newenvironment{exercise}[2][Exercise]{\begin{trivlist}
\item[\hskip \labelsep {\bfseries #1}\hskip \labelsep {\bfseries #2.}]}{\end{trivlist}}
\newenvironment{problem}[2][Problem]{\begin{trivlist}
\item[\hskip \labelsep {\bfseries #1}\hskip \labelsep {\bfseries #2.}]}{\end{trivlist}}
\newenvironment{question}[2][Question]{\begin{trivlist}
\item[\hskip \labelsep {\bfseries #1}\hskip \labelsep {\bfseries #2.}]}{\end{trivlist}}
\newenvironment{corollary}[2][Corollary]{\begin{trivlist}
\item[\hskip \labelsep {\bfseries #1}\hskip \labelsep {\bfseries #2.}]}{\end{trivlist}}
 
\begin{document}
 
% --------------------------------------------------------------
%                         Start here
% --------------------------------------------------------------
 
\title{Assignment 8}%replace X with the appropriate number
\author{Md Ali\\ %replace with your name
Abstract Algebra MATH 4307} %if necessary, replace with your course title
 
\maketitle
 
\begin{problem}{2.1.8} %You can use theorem, exercise, problem, or question here.  Modify x.yz to be whatever number you are proving
If $G$ is an abelian group, prove that $(a*b)^n=a^n*b^n$ for all integers $n$.
\end{problem}
 
\begin{proof}
We will use proof by induction \\ \\
%Note 1: The * tells LaTeX not to number the lines.  If you remove the *, be sure to remove it below, too.
%Note 2: Inside the align environment, you do not want to use $-signs.  The reason for this is that this is already a math environment. This is why we have to include \text{} around any text inside the align environment.
Let's look at $n=1$ \\ \\
\centerline{It is trivial to see that $(a*b)^n=a*b=a^n*a^n$} \\ \\
Now let's look at the inductive step, let's suppose that $(a*b)^n=a^n*b^n$ for some $n \geq 1$. \\ \\
\centerline{$(a*b)^{n+1}=(a*b)^n*(a*b) = a^n*b^n*a*b = a^n*a*b^n*b=a^{n+1}*b^{n+1}$} \\ \\
\centerline{Hence from above we can see that $G$ is abelian by induction, and $(a*b)^n=a^n*b^n$ for all positive integers $n$.} \\ \\
Now let's take a look at the case where $n=0$ \\ \\
\centerline{It is obvious to see that $(a*b)^n=i=a^n*b^n$ where $i$ is the identity element of $G$.} \\ \\
Now let's suppose that $n$ is a negative integer. \\ \\
\centerline{$(a*b)^{n}=(a*b)^{-(n)} = [(a*b)^{-n}]^{-1} = [a^{-n}*b^{-n}]^{-1}=b^{-(-n)}*a^{-(n)}=b^n*a^n=a^n*b^n$} \\ \\
\centerline{Without loss of generality we can see that using proof by induction as in earlier in the problem }
\centerline{we can see that $-n$ is a positive integer.} \\ \\
\centerline{\fbox{Hence $(a*b)^n=a^n*b^n$ $\forall$ integers $n$.}}
\end{proof}

\begin{problem}{2.1.9}
If $G$ is a group in which $a^2 =e$ for all $a \in G$, show that $G$ is abelian.
\end{problem}

\begin{proof}
Let's multiple both sides of $a^2 = e$ by $a^{-1}$ \\ \\
\centerline{$a^2 = e \rightarrow a^2 *a^{-1} = e * a^{-1} \rightarrow a = a^{-1}$} \\ \\
\centerline{Hence from above every element of $G$ is its own inverse} \\ \\
Now let $a,b \in G$ \\ \\
\centerline{$a*b = (a*b)^{-1}=b^{-1}*a^{-1}=b*a$} \\ \\

\centerline{\fbox{Hence from above $G$ is abelian}}
\end{proof}

\begin{problem}{2.1.13}
Show that any group of order 4 or less is abelian.
\end{problem}


\begin{proof}
Suppose that $G$ is any group $|G| \leq 6$. We will prove that $G$ is abelian. \\ \\
\centerline{If $|G|=1$, then $G=\{i\}$. This is trivial because it only contains the identity, hence abelian} \\ \\
Now, we will prove that every group of prime order is cyclic which makes it abelian.  \\ \\
\centerline{Suppose that $|G| = p$, a prime. Let $g \in G \ni g \neq i$, the identity.}
\centerline{Then we can see that $g \neq 1$}\\ \\
\centerline{By the Lagrange theorem order of an element of a group divides order of a group}
\centerline{i.e. $O(g)$ must divide $p=O(G)=|G|$} \\ \\ 
\centerline{$p$, being a prime has only 1 and $p$ as its divisors}
\centerline{Hence we get that $O(g) = p$ and $O(g) \neq 1$}
\centerline{Thus we get that $O(G) = O(g) = p$ and $G = <g>$} \\ \\
This shows that $G$ is indeed a cyclic group generated by $g$ and since every cyclic group is abelian, $G$ is abelian as well! \\ \\
\centerline{This, if $|G| = 2$ or $3$, $G$ is abelian, since 2 and 3 are prime numbers.} \\ \\
Finally, let's look at when $|G| = 4$ \\ \\
\centerline{$|G|=4 =2^2=$prime square}
\centerline{We can conclude that for any prime $p$, every group of order $p^2$ is abelian}
\centerline{Hence, every group of order $4(=2^2)$ is abelian}
\centerline{In essence, there are only 2 non-isomorphic abelian groups of order 4}
\centerline{One is isomorphic to $\mathbb{Z}_4$ under addition $\mod 4$}
\centerline{The other is isomorphic to the non-cyclic group, but hence both of are abelian} \\ \\
\centerline{\fbox{Hence $\forall 4$ cases, $G$ is abelian}} \\ \\
\end{proof}
 
 \begin{problem}{2.1.14}
If $G$ is any group and $a,b,c \in G$, show that if $a*b=a*c$,then $b=c$, and if $b*a=c*a$, then $b=c$.
\end{problem}

\begin{proof}
Suppose $a,b,c \in G \ni a*b=a*c$. Let $i$ denote the identity in $G$. So $G$ is a group, $a^{-1} \in G$. Multiplying both sides by $a^{-1}$ \\ \\
\centerline{$a^{-1}*(a*b)=a^{-1}*(a*c)$}
\centerline{$(a^{-1}*a)*b=(a^{-1}*a)*c$}
\centerline{$i*b=i*c \rightarrow b=c$} \\ \\
We can see from above that if $a*b=a*c$, then $b=c$. Without loss of generality, this works for $b*a=c*a$ as well so if $b*a=c*a$, then $b=c$. \\ \\
\centerline{\fbox{Hence if $a*b=a*c$,then $b=c$, and if $b*a=c*a$, then $b=c$}} \\ \\
\end{proof}

\begin{problem}{2.1.15}
Express $(a*b)^{-1}$ in terms of $a^{-1}$ and $b^{-1}$
\end{problem}

\begin{proof}
Let $a$ and $b$ be elements of a group $G$. \\ \\
\centerline{Let's take $(a*b)*(b^{-1}*a^{-1})=((a*b)*b^{-1})*a^{-1}$}
\centerline{$\rightarrow (a*(b*b^{-1}))*a^{-1}$}
\centerline{$(a*i)*a^{-1}$}
\centerline{$a*a^{-1}$}
\centerline{$=i$} \\ \\
\centerline{Let's take $(b^{-1}*a^{-1})*(a*b)=b^{-1}*(a^{-1}*(a*b))$}
\centerline{$\rightarrow b^{-1}*((a^{-1}*a)*b)$}
\centerline{$b^{-1}*(i*b)$}
\centerline{$b^{^-1}*b$}
\centerline{$=i$} \\ \\
\centerline{\fbox{Hence from above we achieve $(a*b)^{-1} = b^{-1}*a^{-1}$}} \\ \\

\end{proof}

\begin{problem}{2.1.16}
Using the result of Problem 15, prove that a group $G$ in which $a = a^{-1}$ for every $a \in G$ must be abelian.
\end{problem}

\begin{proof}
Using what we know from Problem 15 and let $a,b \in G$ we can see that \\ \\
\centerline{$a*b = (a*b)^-1=b^{-1}*a^{-1}=b*a$} \\ \\
\centerline{\fbox{Hence we can see rom above that they are all equal to their own inverses.}}

\end{proof}

% --------------------------------------------------------------
%     You don't have to mess with anything below this line.
% --------------------------------------------------------------
 
\end{document}