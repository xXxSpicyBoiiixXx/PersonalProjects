% --------------------------------------------------------------
% This is all preamble stuff that you don't have to worry about.
% Head down to where it says "Start here"
% --------------------------------------------------------------
 
\documentclass[12pt]{article}
 
\usepackage[margin=1in]{geometry} 
\usepackage{amsmath,amsthm,amssymb}
 
\newcommand{\N}{\mathbb{N}}
\newcommand{\Z}{\mathbb{Z}}
 
\newenvironment{theorem}[2][Theorem]{\begin{trivlist}
\item[\hskip \labelsep {\bfseries #1}\hskip \labelsep {\bfseries #2.}]}{\end{trivlist}}
\newenvironment{lemma}[2][Lemma]{\begin{trivlist}
\item[\hskip \labelsep {\bfseries #1}\hskip \labelsep {\bfseries #2.}]}{\end{trivlist}}
\newenvironment{exercise}[2][Exercise]{\begin{trivlist}
\item[\hskip \labelsep {\bfseries #1}\hskip \labelsep {\bfseries #2.}]}{\end{trivlist}}
\newenvironment{problem}[2][Problem]{\begin{trivlist}
\item[\hskip \labelsep {\bfseries #1}\hskip \labelsep {\bfseries #2.}]}{\end{trivlist}}
\newenvironment{question}[2][Question]{\begin{trivlist}
\item[\hskip \labelsep {\bfseries #1}\hskip \labelsep {\bfseries #2.}]}{\end{trivlist}}
\newenvironment{corollary}[2][Corollary]{\begin{trivlist}
\item[\hskip \labelsep {\bfseries #1}\hskip \labelsep {\bfseries #2.}]}{\end{trivlist}}
 
\begin{document}
 
% --------------------------------------------------------------
%                         Start here
% --------------------------------------------------------------
 
\title{Assignment 21}%replace X with the appropriate number
\author{Md Ali\\ %replace with your name
Abstract Algebra MATH 4307} %if necessary, replace with your course title
 
\maketitle
 
\begin{problem}{4.1.1} %You can use theorem, exercise, problem, or question here.  Modify x.yz to be whatever number you are proving
Find all the elements in $\mathbb{Z}_{24}$ that are invertible in $\mathbb{Z}_24$
\end{problem}
 
\begin{proof}
We know that $\mathbb{Z}_{24}$ elements are $\mathbb{Z} = \{ [0],[1],...,[23] \}$.\\
We also know that zero nor a zero divisor can be invertible. In turn we get \\ \\
\centerline{ $\{1,5,7,11,13,17,19,23\}$} \\ \\

Hence we can see that the elements $[n]$ of $\mathbb{Z}$ which are invertible have the form of an a $gcd(n,24)=1$ and those are \\ \\
\centerline{\fbox{$ [1],[5],[7],[11],[13],[17],[19],[23]$}} \\ \\
\end{proof}

\begin{problem}{4.1.2}
Show that any field is an integral domain. 
\end{problem}

\begin{proof}
Let $F$ be a field. By definition, $F$ is commutative ring, so to show that $F$ is an integral domain we just need to show that it has no zero divisors. \\ \\
Suppose $a,b \in F$ and $ab =0$ where $a \neq 0$. Since $F$ is a field than there exist $a^{-1} \in F$ this \\ \\
\centerline{$ab=0$} 
\centerline{$a^{-1}(ab)=a^{-1}0$}
\centerline{$(a^{-1}a)b=0$}
\centerline{$ib=0$}
\centerline{$b=0$}\\ \\
Since $a$ was an arbitrary nonzero elements of $F$, it follows that there is no zero divisors in $F$ \\ \\
\centerline{\fbox{Hence, all fields are integral domains}} \\ \\
\end{proof}

\begin{problem}{4.1.3}
Show that $\mathbb{Z}_n$ is a field if and only if $n$ is a prime.
\end{problem}

\begin{proof}
Suppose that $n$ is not prime and consider the ring $\mathbb{Z}_n$ \\ 
Then there exists positive integers $1<k$,$m<n$ such that $n=km$, but then $[k][m]=[n]=0$ hence there are zero divisors in $\mathbb{Z}_n$. This menas that $\mathbb{Z}_n$ cannot be a field. \\ \\
\centerline{\fbox{Hence, by contradiction $n$ must be prime!}}
\end{proof}
 
\begin{problem}{4.1.4}
Verify that Example 8 is a ring. Find all its invertible elements. 
\end{problem}

\begin{proof} 
Let $\mathbb{Q}_{odd} = \{ \dfrac{p}{q} | p,q \in Z, gcd(p,q)=1, q\neq 0$, and q is odd. \\ \\
Also $\dfrac{0}{1} \in \mathbb{Q}_{odd}$ and $\dfrac{1}{1} \in \mathbb{Q}_{odd}$ and $\mathbb{Q} \neq \emptyset$ \\ \\
Let $\dfrac{p_1}{q_1}, \dfrac{p_2}{q_2} \in \mathbb{Q}_{odd}$ then we achieve that \\ \\
$\dfrac{p_1}{q_1} -  \dfrac{p_2}{q_2} = \dfrac{p_1q_2-p_2q_1}{q_1q_2} \in \mathbb{Q}_{odd}$ \\ \\
This makes it where $\dfrac{p}{q}$ is invertible in $\mathbb{Q}_{odd}$ if and only if $p$ is odd. \\ \\
\centerline{\fbox{Hence all the invertible elements are only when $p$ is odd}} \\ \\
\end{proof}

\begin{problem}{4.1.7}
Work out the following \\ 
(a)
 $$\begin{pmatrix}
	1 & 2\\
	4 & -7
	\end{pmatrix}
	\begin{pmatrix}
	\dfrac{1}{5} & \dfrac{2}{3} \\
	0 & 1
	\end{pmatrix}
	$$
\bigskip
(b)
 $$\begin{pmatrix}
	1 & 1\\
	1 & 1
	\end{pmatrix}^2
	$$
	\bigskip
	(c)
	 $$\begin{pmatrix}
	\dfrac{1}{2} & \dfrac{1}{2}\\
	0 & 0
	\end{pmatrix}^3
	$$
	\bigskip
	(d)
	 $$\begin{pmatrix}
	a & b\\
	c & d
	\end{pmatrix}
	\begin{pmatrix}
	1 & 0 \\
	0 & 0
	\end{pmatrix}
	-
	\begin{pmatrix}
	1 & 0 \\
	0 & 0
	\end{pmatrix}
	\begin{pmatrix}
	a & b\\
	c & d
	\end{pmatrix}
	$$
\end{problem}

\begin{proof} 
All of these are done by using simply matrix algebra. I will not type all my work and simply put the outcomes of each. \\ \\
(a)  $$\begin{pmatrix}
	\dfrac{1}{5} & \dfrac{8}{3}\\
	\dfrac{4}{5} & \dfrac{-13}{3}
	\end{pmatrix}
	$$
	\bigskip
	(b)
	$$\begin{pmatrix}
	2 & 2\\
	2& 2
	\end{pmatrix}
	$$
	\bigskip
	(c)
	$$\begin{pmatrix}
	\dfrac{1}{8} & \dfrac{1}{8}\\
	0 & 0
	\end{pmatrix}
	$$
	\bigskip
	(d)
	$$\begin{pmatrix}
	0 & -b\\
	c & 0
	\end{pmatrix}
	$$
	\bigskip
\end{proof}

\begin{problem}{4.1.8}
Find all the matrices $$\begin{pmatrix}
	a & b\\
	c & d
	\end{pmatrix}
	$$
	such that 
	$$\begin{pmatrix}
	a & b\\
	c & d
	\end{pmatrix}
	\begin{pmatrix}
	1 & 0\\
	0 & 0
	\end{pmatrix}
	=
	\begin{pmatrix}
	1 & 0\\
	0 & 0
	\end{pmatrix}
	\begin{pmatrix}
	a & b\\
	c & d
	\end{pmatrix}
	$$
\end{problem}

\begin{proof} 
From the previous problem we know that this will end up being 
$$\begin{pmatrix}
	a & 0\\
	c & 0
	\end{pmatrix}
	=
	\begin{pmatrix}
	a & b\\
	0 & 0
	\end{pmatrix}
	$$
	We can see that b=c=0. Hence the set of all such matrices that satisfy the condition above is 
	$$\begin{pmatrix}
	a & 0\\
	0 & d
	\end{pmatrix}
	$$
	Where $a,d$ are arbitrary real numbers. \\ \\
\end{proof}

\begin{problem}{4.1.9}
Find all $2 \times 2$ matrices 
$$ \begin{pmatrix}
a &b \\
c &d 
\end{pmatrix}
$$ 
that commute with all $2 \times 2$ matrices. \\ \\
\end{problem}

\begin{proof}
From problem 8 we have that 
$$\begin{pmatrix}
	a & 0\\
	0 & d
	\end{pmatrix}
	$$
	
From here we can create an arbitrary $2 \times 2$ matrix. 
$$\begin{pmatrix}
	a & 0\\
	0 & d
	\end{pmatrix}
	\begin{pmatrix}
	p & q \\
	r & s
	\end{pmatrix}
	=
	\begin{pmatrix}
	p & q \\
	r & s
	\end{pmatrix}
	\begin{pmatrix}
	a & 0\\
	0 & d
	\end{pmatrix}
	$$
	This equals 
	$$\begin{pmatrix}
	ap & aq\\
	dr & ds
	\end{pmatrix}
	=
	\begin{pmatrix}
	ap & dq \\
	ar & ds
	\end{pmatrix}
	$$
	We can see that $a=d$. Hence all matrix of the forms 
	$$
	\begin{pmatrix}
	a & 0 \\
	0 &a
	\end{pmatrix}
	$$
	commutes with ever $2 \times 2$ matrices because 
	$$ 
	\begin{pmatrix}
	a & 0 \\
	0 & a
	\end{pmatrix}
	=
	a \begin{pmatrix}
	1 & 0 \\
	0 & 1
	\end{pmatrix}
	$$ 
	Where $a$ is some scalar and identity matrix commutes with every other matrix. \\ \\
\end{proof}


% --------------------------------------------------------------
%     You don't have to mess with anything below this line.
\end{document}