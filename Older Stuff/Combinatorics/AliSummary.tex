%% ****** Start of file template.aps ****** %
%%
%%
%%   This file is part of the APS files in the REVTeX 4 distribution.
%%   Version 4.0 of REVTeX, August 2001
%%
%%
%%   Copyright (c) 2001 The American Physical Society.
%%
%%   See the REVTeX 4 README file for restrictions and more information.
%%
%
% This is a template for producing manuscripts for use with REVTEX 4.0
% Copy this file to another name and then work on that file.
% That way, you always have this original template file to use.
%
% Group addresses by affiliation; use superscriptaddress for long
% author lists, or if there are many overlapping affiliations.
% For Phys. Rev. appearance, change preprint to twocolumn.
% Choose pra, prb, prc, prd, pre, prl, prstab, or rmp for journal
%  Add 'draft' option to mark overfull boxes with black boxes
%  Add 'showpacs' option to make PACS codes appear
\documentclass[aps,prl,twocolumn,showpacs,superscriptaddress,groupedaddress]{revtex4}  % for review and submission
%\documentclass[aps,preprint,showpacs,superscriptaddress,groupedaddress]{revtex4}  % for double-spaced preprint
\usepackage{graphicx}  % needed for figures
\usepackage{dcolumn}   % needed for some tables
\usepackage{bm}        % for math
\usepackage{amssymb}   % for math

% avoids incorrect hyphenation, added Nov/08 by SSR
\hyphenation{ALPGEN}
\hyphenation{EVTGEN}
\hyphenation{PYTHIA}


\begin{document}

% The following information is for internal review, please remove them for submission
\widetext
\leftline{Version 01 as of \today}
\leftline{Primary Authors: Md H. I. Ali}
\leftline{Additional Investigators: N/A}
\leftline{To be submitted to Department of Mathematics, Texas State University}
\bigskip
\centerline{\em INTERNAL DOCUMENT -- NOT FOR PUBLIC DISTRIBUTION}

% the following line is for submission, including submission to the arXiv!!
%\hspace{5.2in} \mbox{Fermilab-Pub-04/xxx-E}

\title{Paper Summary}     % D0 authors (remove the first 3 lines
                             % of this file prior to submission, they
                             % contain a time stamp for the authorlist)
                             % (includes institutions and visitors)
\date{\today}


\begin{abstract}
In this paper we will be summarizing "New Direction in Enumerative Chess Problems" by Noam D. Elkies. In an enumerative chess problem, the set of moves in the solution is usually unique but the order is not. The task is to count the permutations via na isomorphic problem in enumerative combinatorics. 
\end{abstract}

\pacs{420}
\maketitle

%\section{\label{sec:level1}First-level heading}
% sections are not used for PRL papers
\section{\label{sec: level1}I. Overall Mathematical Contribution}
The overall mathematical contribution is that the pieces in chess can be arranged in a way where we can have interesting numbers as well as an active field in combinatorial games research. The essential question that is answer is how many ways can we get from position X to position Y. The paper give 14 well detailed examples of this from the beginning of combinatoric problems in chess to what is actively being looked at now. There are some interesting numbers that come up as well as how we can formulate numbers ourselves as the author has shown that to honor Richard Stanley's 60th birthday. 

%\subsection{\label{sec:level2}Second-level heading: Formatting}
% subsections are not used for PRL papers
\section{\label{sec: level2}II. Terminology and Problems}
The author goes over various problems in the paper and utilizes various chess terms to explain the problems as well as different math sequences. In addition, one such called Euler's number as a generating function, which I have read is essential in many combinatorial problems as a whole. The terms he uses for chess are series helpmate, name in n, and shortest games. 

\section{\label{sec: level3}III. Main Theorem}
There seems to be no theorems used since this is a paper about new direction towards enumerative chess problems. The only that we can see is the patterns and the way the pieces are set up. From this we can determine the method to utilize to find the number of solutions or the shortest paths to take. Hence there are many numbers and ways in the paper depending on the way the pieces are set up and the question being asked. \\ \\

\section{\label{sec: level4}IV. Applications of the Main Theorem} 
This has helped to write many chess problem solving books and inspires the game itself to continue moving forward. I have personally bought a problem solving chess book due to this project and it has rekindled my love for the game. I believe that the direct impact it has that it can bring joy to people to figure out these puzzles as well as how math can be used to discover the different ways someone can lose or win is quite comical and a new way to look at paths! 

\section{\label{sec: level5}V. Impact in Combinatorics}
The impact this paper has on combinatorics is that instead of having the pieces staying stationary now, which was done in the past to fit easily in combinatorial problem, now we have a dynamic struggle between player 1 and player 2, which adds elements to asking how many ways can we get from position X to position Y in a very difficult manner. As in the paper we can see that there is a certain game set up where the solution is one million shortest games, which is in fact a lot of different solutions for a chess game to configure into. With this knowledge we can even have a series of solution such as Fibonacci numbers and even certain type of numbers such as the Catalan and Euler numbers. Overall I feel that this paper would peak the chess community heavily. 
\end{document}
%
% ****** End of file template.aps ******