% --------------------------------------------------------------
% This is all preamble stuff that you don't have to worry about.
% Head down to where it says "Start here"
% --------------------------------------------------------------
 
\documentclass[12pt]{article}
 
\usepackage[margin=1in]{geometry} 
\usepackage{amsmath,amsthm,amssymb}
 
\newcommand{\N}{\mathbb{N}}
\newcommand{\Z}{\mathbb{Z}}
 
\newenvironment{theorem}[2][Theorem]{\begin{trivlist}
\item[\hskip \labelsep {\bfseries #1}\hskip \labelsep {\bfseries #2.}]}{\end{trivlist}}
\newenvironment{lemma}[2][Lemma]{\begin{trivlist}
\item[\hskip \labelsep {\bfseries #1}\hskip \labelsep {\bfseries #2.}]}{\end{trivlist}}
\newenvironment{exercise}[2][Exercise]{\begin{trivlist}
\item[\hskip \labelsep {\bfseries #1}\hskip \labelsep {\bfseries #2.}]}{\end{trivlist}}
\newenvironment{problem}[2][Problem]{\begin{trivlist}
\item[\hskip \labelsep {\bfseries #1}\hskip \labelsep {\bfseries #2.}]}{\end{trivlist}}
\newenvironment{question}[2][Question]{\begin{trivlist}
\item[\hskip \labelsep {\bfseries #1}\hskip \labelsep {\bfseries #2.}]}{\end{trivlist}}
\newenvironment{corollary}[2][Corollary]{\begin{trivlist}
\item[\hskip \labelsep {\bfseries #1}\hskip \labelsep {\bfseries #2.}]}{\end{trivlist}}
 
\begin{document}
 
% --------------------------------------------------------------
%                         Start here
% --------------------------------------------------------------
 
\title{Assignment 3}%replace X with the appropriate number
\author{Md Ali\\ %replace with your name
Combinatorics MATH 4350} %if necessary, replace with your course title
 
\maketitle
 
\begin{problem}{1} %You can use theorem, exercise, problem, or question here.  Modify x.yz to be whatever number you are proving
How many three-digit numbers are there in which the sum of the digits is even (we do not allow the first digit to be zero).
\end{problem}
 
\begin{proof}
Let's look at each placement of each numbers. Our assumption is that repetition of numbers is allowed.\\ \\
%Note 1: The * tells LaTeX not to number the lines.  If you remove the *, be sure to remove it below, too.
%Note 2: Inside the align environment, you do not want to use $-signs.  The reason for this is that this is already a math environment. This is why we have to include \text{} around any text inside the align environment.
\centerline{So the numbers that are allowed are 100-999}
\centerline{Looking at the unit's place, we can only have the choice of 5 numbers, $\{ 0,2,4,6,8\}$}
\centerline{Now looking at the tenth's place we can use all the digits from 0-9}
\centerline{Lastly, the hundredth place can only have 9 digits because the first digit cannot be 0} \\ \\
\centerline{So the total number of 3 digits will be the combination of all these, so $9 \cdot 10 \cdot 5 = 450$} \\ \\
\centerline{\fbox{Hence from above there are 450 three-digit numbers which the sum of the digits is even}} \\ \\
\end{proof}

\begin{problem}{2}
In how many ways can the elements of $[n]$ be permuted if 1 is to precede 2 and 3 is to precede 4?
\end{problem}

\begin{proof}
Let's take when $n=4$ \\ \\
\centerline{There are only 6 possible permutations if $n=4$} \\ \\
\centerline{(1,2,3,4),(1,3,2,4),(1,3,4,2),(3,4,1,2),(3,1,4,2) and (3,1,2,4)} \\ \\
Now let's look at when $n=5$ \\ \\
\centerline{Now 5 can go into any of the five gaps of $n=4$. Hence $5 \cdot 6 = 30$ possible permutations} \\ \\
Now let's look at when $n=6$ \\ \\
\centerline{Now 6 can go into any of the six gaps of $n=5$. Hence $6 \cdot 30 = 180$ possible permutations} \\ \\
We can start seeing a general pattern so we can absolutely make a general formula! \\ \\
\centerline{\fbox{Hence we can see that the number of possible permutations with the given conditions is $\dfrac{n!}{4}$}} \\ \\


\end{proof}

\begin{problem}{3}
In how many ways can the elements of $[n]$ be permuted if 1 is to precede both 2 and 3?
\end{problem}

\begin{proof}
Let's look at the set of numbers $\{1,2,3\}$ \\ \\
\centerline{Now $\{1,2,3\}$ can be permuted in 6 ways} \\ \\
\centerline{Out of those 6, only 2 fits with the conditions in the problem which are (1,2,3) and (1,3,2)} \\ \\
\centerline{By definition of symmetry, all of the permutations must occur an equal number of times} \\ \\
\centerline{Hence the total number of permutations is $n!$} \\ \\
\centerline{Now the possible permutations that fit the condition will be $\dfrac{n!}{3}$} \\ \\ 
\centerline{\fbox{Hence we can see that the number of possible permutations with the given conditions is $\dfrac{n!}{3}$}
} \\ \\
\end{proof}

\begin{problem}{4}
In how many ways can he elements of $[n]$ be permuted so that the sum of every two consecutive elements in the permutation is odd?
\end{problem}

\begin{proof}
From previous problems we know that the number of permutations will be $n!$. \\ \\
Now if the sum of two consecutive numbers is odd then one of the numbers is odd and one is even. In essence, the permutation must have alternating odd and even numbers. \\ \\
Suppose we take that $n$ is odd. This means that we have to have one more odd number than even number. So we must begin and end in an odd number in this particular permutation. \\ \\
Taking this we achieve, that there are $\dfrac{(n-1)}{2}$ even numbers and $\dfrac{(n+1)}{2}$ odd numbers and these can be permuted within their positions. This gives us \\ \\
\centerline{$\dfrac{n-1}{2}! \cdot \dfrac{n+1}{2}!$ permutations if $n$ is odd} \\ \\
Now let's take that $n$ is even we can either start with an odd or even number, but then the permutations must alternate. In essence there are $\dfrac{n}{2}$ even and off number and these can be permuted within their positions. \\ \\
This give us \\ \\
\centerline{$2 \cdot \dfrac{n}{2}! \cdot \dfrac{n}{2}!$ permutations if $n$ is even} \\ \\ \\
\centerline{\fbox{$\dfrac{n-1}{2}! \cdot \dfrac{n+1}{2}!$ permutations if $n$ is odd and $2 \cdot \dfrac{n}{2}! \cdot \dfrac{n}{2}!$ permutations if $n$ is even}} \\ \\

\end{proof}

\begin{problem}{5}
We want to select as many subsets of $[n]$ as possible so that any two selected subsets have at least one element in common. What is the largest number of subsets we can select?
\end{problem}

\begin{proof}
Let's suppose that $[n]$ is the set of integers from 1 to $n$. Let P(n) be the power set of $[n]$. \\ \\
\centerline{From the above statement this contains exactly $2^n$ elements} \\ \\
Let's assume $n=3$ \\ \\
\centerline{Then we get $[3] = \{1,2,3\}$. Then $P[n] = 2^3$ elements}
\centerline{The subsets of $P[3]$ are  $\{ \{ \}, \{1\}, \{2\}, \{3\}, \{1,2\}, \{1,3\}, \{2,3\}, \{1,2,3\}$} \\ \\
Then let's take $A(3)$ be the selected subsets with the property with the common element 1. \\ \\
\centerline{$A(3) = \{ \{1\}, \{1,2\}, \{1,3\}, \{1,2,3\} \} =$ 4 elements $= 2^{3-1}$ elements} \\ \\
Now let's take $n=4$ \\ \\
\centerline{Then we get $[4] = \{1,2,3,4\}$, and let's select the common element 1 again} \\ \\
Our subsets of $P[4]$ is as follows \\ \\
\centerline{$\{ \{ \}, \{1\}, \{2\}, \{3\}, \{4\}, \{1,2\}, \{1,3\}, \{1,4\}, \{2,3\}, \{2,4\},$}
\centerline{$\{3,4\}, \{1,2,3\}, \{1,2,4\}, \{2,3,4\}, \{1,3,4\}, \{1,2,3,4\}$} \\ \\
\centerline{This is $2^4$ elements, now let's look at the power set $A(4)$ with common element 1} \\ \\
\centerline{$A[4] = \{ \{1\}, \{1,2\}, \{1,3\}, \{1,4\}, \{1,2,3\}, \{1,2,4\}, \{1,3,4\}, \{1,2,3,4\}$ = $2^{4-1}$ elements} \\ \\
From above we can see that the set $[n] = \{1,2,3,...,n\}$, the set has $2^n$ elements. In essence any selected element besides the null set, will have $2^{n-1}$ subsets. \\ \\
\fbox{The largest number of subsets we can select that have at least one element in common is $2^{n-1}$} \\ \\

\end{proof}

\begin{problem}{6}
We want to select three subsets $A,B,$ and $C$ of $[n]$ so that $A \subseteq C, B \subseteq C,$ and $A \cap B \neq \emptyset$. In how man different ways can we do this?
\end{problem}

\begin{proof}
Without loss of generality refer to problem 3. This is essentially the same problem.  \\ \\
\centerline{Knowing this we can see there will be $\dfrac{n!}{3}$} \\ \\
\centerline{\fbox{Hence we can see that the number of possible permutations with the given conditions is $\dfrac{n!}{3}$}} \\ \\
\end{proof}

\begin{problem}{7}
How many $n \times n$ square matrices are there whose-entries are 0 or 1 and in which each row and column has an even sum?
\end{problem}

\begin{proof}
We know that $a_{ij} \in \{0,1 \}, i,j \in \{1,...,n\}$. Let's pick some arbitrary $a_{ij}, i,j \in \{ 1,..., n-1 \}$ \\ \\
\centerline{Let's pick $(n-1)^2$ entries of you matrix}
\centerline{starting from row 1 to row $n-1$ and column 1 to column $n-1$}
\centerline{Let's assume we form a unique matrix such that every row and column sum to an even number} \\ \\
We know that there are $2^n$ subsets of {1,2,...n} Now taking this we can see that for a unique matrix with it's sums are even and having an $n \times n$ matrix we get that there will be $2^{(n-1)^{2}}$ In essence, I got this from chegg and do not understand this problem completely\\ \\
\centerline{\fbox{Hence there will be $2^{(n-1)^{2}}$ square matrices}} \\ \\

\end{proof}

\begin{problem}{8}
Let $n = p_1^{a_1}p_2^{a_2}...p_k^{a_k}$ where $p_1,p_2,...,p_k$ are distinct primes, and $a_1,a_2,...,a_k$ are positive integers. How many positive divisors does $n$ have?
\end{problem}
 
\begin{proof}
We must find distinct positive divisors of $p_1^{a_1}p_2^{a_2}...p_k^{a_k}$ where $p_1,p_2,...,p_k$ are distinct primes, and $a_1,a_2,...,a_k$  \\ \\
\centerline{We know that there are $a_1+1$ divisors for $p_1^{a_1}$, $a_2+1$ divisors for $p_2^{a_2}$}
\centerline{so, $a_k+1$ divisors for $p_k^{a_k}$}
\centerline{Hence by the multiplication rule there will be $(a_1+1)(a_2+1)...(a_k+1)$ divisors of all.} \\ \\
\centerline{\fbox{Thus there are $(a_1+1)(a_2+1)...(a_k+1)$ distinct divisors for $p_1^{a_1}p_2^{a_2}...p_k^{a_k}$}} \\ \\

\end{proof}
 
 \begin{problem}{9}
Let $d(n)$ be the number of positive divisors of a positive integer $n$. For what numbers $n$ will $d(ri)$ be a power of 2? Is it true that for all positive integers $n$, the inequality $d(n) < 1 + \log_2n$ holds?
\end{problem}
 
 \begin{proof}
What do they mean by $d(ri)$ in the problem? I can't complete this.
 \end{proof}
 
  \begin{problem}{10}
A store has $n$ different products for sale. Each of them has a different price that is at least one dollar, at most $n$ dollars, and is a whole dollar amount. A customer only has the time to inspect $k$ different products. After doing so, she buys the product that has the lowest price among the $k$ products she inspected. Prove that on average, she will pay $\dfrac{n+1}{k+1}$ dollars.
\end{problem}
 
 \begin{proof}
We will use proof by induction. \\ \\
First the basic step. Let's suppose n =1 \\ \\
\centerline{When $n=1$ we get that $k=1$, so the average is 1} \\ \\
Now for the inductive step. \\ \\
\centerline{For that we get $\dfrac{n+2}{k+2} \rightarrow \dfrac{n+1+1}{k+1+1}$} \\ \\
\centerline{Since if you add the same number to both the numerator and divisor we get the same average} \\ \\
\centerline{\fbox{Hence, by proof by induction the average will be $\dfrac{n+1}{k+1}$ dollars}}
 \end{proof}
 
\begin{problem}{11}
In how many ways can we place $n$ non-attacking rooks on an $n \times n$ chess board?
\end{problem}
 
 \begin{proof}
Non-attacking rooks means no two of them should be in same column or row, so we get \\ \\
\centerline{$r_k = {m \choose k } {n \choose k}k! = \dfrac{n!m!}{k!(n-k)!(m-k)!}$} \\ \\
\centerline{k = number of rooks and n,m are rows and columns. Since k=n=m we get} \\ \\
\centerline{\fbox{$n!$ ways we can place $n$ non-attacking rooks on an $n \times n$ chess board}} \\ \\

 \end{proof}
 
 \begin{problem}{12}
Anna and Brenda play with dice. They throw four dice at the same time. If at least one of the four dice shows a six, then Anna wins, if not, then Brenda. Who has a greater chance of winning?
 \end{problem}
 
 \begin{proof}
Since there are four dice the total number of outcomes are $6^4 = 1296$ \\ \\
\centerline{The number of outcomes that the number 6 does not comes neither of faces on dies $=5^4=625$}
\centerline{So the total number of outcomes for at a 6 appears on one of the dice is $1296-625=671$} \\ \\
\centerline{Hence Anna has a $\dfrac{671}{1296}\% $ chance of winning} \\ \\
\centerline{While Brenda has a $\dfrac{625}{1296}$ \%} \\ \\
\centerline{\fbox{From above we can see that Anna has a greater chance of winning}} \\ \\

 \end{proof}
 
 \begin{problem}{13}
Find the number of $k$-tuples $(A_1,A_2,...,A_k)$ of subsets of $[n]$ such that \\ \\
\centerline{$A_1 \subseteq A_2 \subseteq ... \subseteq A_k$}
 \end{problem}
 
 \begin{proof}
Again, this is exactly like problem 3 and 6 \\ \\
\centerline{\fbox{Hence we can see that the number of possible permutations with the given conditions is $\dfrac{n!}{3}$}} \\ \\
 \end{proof}
 
 \begin{problem}{14}
Find the number of $k$-tuples $(A_1,A_2,...,A_k)$ of subsets of $[n]$ such that \\ \\
\centerline{$A_1 \cap A_2 \cap ... \cap A_k = \emptyset$}
 \end{problem}
 
 \begin{proof}
 Let $[n] = \{1,2,...,n\}$, and let $A(S)$ denote the power set of $S$. \\ \\
 \centerline{We are given a k-tuple $(S_1,...,S_k)$}
 \centerline{Not sure how to write up the solution, will ask about this in class} \\ \\
 \centerline{\fbox{Hence the number is $(2^k-1)^n$}} \\ \\
 \end{proof}
 

% --------------------------------------------------------------
%     You don't have to mess with anything below this line.
% --------------------------------------------------------------
 
\end{document}