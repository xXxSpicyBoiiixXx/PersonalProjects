% ****** Start of file apssamp.tex ******
%
%   This file is part of the APS files in the REVTeX 4.1 distribution.
%   Version 4.1r of REVTeX, August 2010
%
%   Copyright (c) 2009, 2010 The American Physical Society.
%
%   See the REVTeX 4 README file for restrictions and more information.
%
% TeX'ing this file requires that you have AMS-LaTeX 2.0 installed
% as well as the rest of the prerequisites for REVTeX 4.1
%
% See the REVTeX 4 README file
% It also requires running BibTeX. The commands are as follows:
%
%  1)  latex apssamp.tex
%  2)  bibtex apssamp
%  3)  latex apssamp.tex
%  4)  latex apssamp.tex
%
\documentclass[%
 reprint,
%superscriptaddress,
%groupedaddress,
%unsortedaddress,
%runinaddress,
%frontmatterverbose, 
%preprint,
%showpacs,preprintnumbers,
%nofootinbib,
%nobibnotes,
%bibnotes,
 amsmath,amssymb,
 aps,
%pra,
%prb,
%rmp,
%prstab,
%prstper,
%floatfix,
]{revtex4-1}

\usepackage{graphicx}% Include figure files
\usepackage{dcolumn}% Align table columns on decimal point
\usepackage{bm}% bold math
%\usepackage{hyperref}% add hypertext capabilities
%\usepackage[mathlines]{lineno}% Enable numbering of text and display math
%\linenumbers\relax % Commence numbering lines

%\usepackage[showframe,%Uncomment any one of the following lines to test 
%%scale=0.7, marginratio={1:1, 2:3}, ignoreall,% default settings
%%text={7in,10in},centering,
%%margin=1.5in,
%%total={6.5in,8.75in}, top=1.2in, left=0.9in, includefoot,
%%height=10in,a5paper,hmargin={3cm,0.8in},
%]{geometry}

\begin{document}

\preprint{APS/123-QED}

\title{Using Spectral Energy Distribution of Alcyone (Eta Tau) to Calculate the Temperature  }% Force line breaks with \\
\thanks{A project, not a journal article}%

\author{Md Ali}
 \email{mha27@txstate.edu}
\affiliation{%
 Physics Department, Texas State University\\
 PHYS 3313 Astrophysics
}%

\collaboration{Spicy Boiiis Inc.}%\noaffiliation

\date{\today}% It is always \today, today,
             %  but any date may be explicitly specified

\begin{abstract}
We are asked to fit the SED (Spectral Energy Distribution) of our favorite star. While SED plots energy versus frequency and is used to characterize different celestial objects in our night's sky. Plotting the SED and fitting the plot with the Planck's function, we can achieve many different parameters from this. The one parameter we are concerned about in this
paper is the temperature of the star.
\end{abstract}

\pacs{Valid PACS appear here}% PACS, the Physics and Astronomy
                             % Classification Scheme.
%\keywords{Suggested keywords}%Use showkeys class option if keyword
                              %display desired
\maketitle

%\tableofcontents

\section{\label{sec:level1}Background}

\subsection{\label{sec:level2}SED and Stellar Evolution}

The SED plot spans several orders of magnitude from various wavelengths, ranging from extreme ultraviolet to far infrared. The majority of every star's lifecycle is spent constructing nuclear fusion. Initially the energy is generated by the fusion of hydrogen atoms at the core of the main sequence star. Later, as the atoms at the core become helium, stars begin to fuse hydrogen along a spherical shell surrounding the core. This causes the star to gradually grow in size. Once a star has exhausted it's nuclear fuel, the core will collapse into a dense celestial object. A star's stellar parameters such as luminosity and temperature remain stable during certain portions of its life. This allows us to organize these stars in a H-R diagram. (fig 1). The H-R diagram plots stars based on luminosity and temperature, analyzing all stars we are able to determine the stellar lifecycles of stars, knowing this we are able to estimate and classify the age of the stars and their current phase in their stellar lifecycle. 

\begin{figure}[htbp] %  figure placement: here, top, bottom, or page
   \centering
   \includegraphics[width=3in]{HR_diagram.jpg} 
   \caption{H-R diagram }
   \label{fig:example}
\end{figure}

\subsection{\label{sec:level3}Alcyone (Eta Tau)}

The star that was investigated was Alcyone (Eta Tau). This star is depicted in fig 2. Alcyone is located in the shoulder blade of Taurus constellation. This is the biggest star of Pleiades or Seven Sisters. This star can be seen with the naked eye due to its brightness. This system is a multi-star system with four stars, with its primary star being Alcyone A and Alcyone B, C, D all orbiting A. 

\begin{figure}[htbp] %  figure placement: here, top, bottom, or page
   \centering
   \includegraphics[width=2in]{Alcyon_star.jpg} 
   \caption{Alcyone}
   \label{fig:example}
\end{figure}

\section{\label{sec:level2}Creating the SED}

Using SIMBAD (Set of Identifications, Measurements, and Bibliography for Astronomical Data) we were able to receive Alcyone's apparent magnitude, this is shown in table 1. The table lists the magnitudes for each particular waveband with the corresponding effective wavelength, which is the central wavelength filer. The magnitudes must be converted into flux densities and the effective wavelength's to orders of frequencies. To convert the magnitudes into flux densities we utilize this equation, $F_v=F_0\cdot10^{(-m/2.5)}$, where $m$ is the apparent magnitude, $F_v$ is the flux density, and $F_0$ is the zero flux conversion factor. All these values are shown in table 1.

\begin{table}[htbp] %  figure placement: here, top, bottom, or page
   \centering
   \includegraphics[width=3in]{Astrophsyics2chart1.jpg} 
   \caption{Characteristics of Alcyone}
   \label{table:example}
\end{table}

\section{\label{sec:level3}Best Fit for Planck's Function}
We will plot the flux density versus wavelength from table 1. We will then use Planck's function, based on wavelength, to create a best fit line. The Planck's function, with respect to lambda and temperature, $B_\lambda(\lambda,T)=\dfrac{2hc^{2}}{\lambda^{5}}\cdot \dfrac{1}{{e^{hc/k_BT\lambda}-1}}$, where $h$ is Planck's constant, $c$ is the speed of light, $k_B$ is Boltzman's constant, $\lambda$ is the wavelength, and $T$ is the temperature. Fig. 3 shows the SED and the Planck's best fit line. 

\begin{figure}[htbp] %  figure placement: here, top, bottom, or page
   \centering
   \includegraphics[width=3in]{AstroPgraph.jpg} 
   \caption{example caption}
   \label{fig:example}
\end{figure}

\section{\label{sec:level4}Conclusion}
In our computer code by printing out the parameters and parameter error we found that the temperature of Alcyone is 9099.19 $\pm$ 1208K. The actual temperature of the star system according to SIMBAD is 12258K, which gives us that our model has a 25.7\% error.

\end{document}
%
% ****** End of file apssamp.tex ******
